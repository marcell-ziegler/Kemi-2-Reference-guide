\documentclass[12pt]{article}
\usepackage[swedish]{babel}
\usepackage[version=4]{mhchem}
\usepackage{amsmath}
\usepackage[swedish]{varioref}
\usepackage{hyperref}
\hypersetup{
    colorlinks=true,
    linkcolor=black
}
\usepackage[swedish]{cleveref}
\usepackage{amsthm}
\usepackage{cancel}
\usepackage{float}
\usepackage{array}
\newtheorem{exm}{Exempel}

\title{Begreppssammanfattning - Kemi 2 \\ Blackebergs Gymnasium}
\author{Marcell Ziegler - NA21D}

\begin{document}
    \begin{titlepage}
        \maketitle
    \end{titlepage}

    \tableofcontents

    \newpage

    \part{Kemisk jämvikt}
    
    En jämvikt är en kemisk reaktion som går åt båda håll med samma reaktionshastighet (lika snabbt). Detta medför att förhållandet mellan reaktanter och produkter förblir densamma. Egentligen är alla reaktioner jämvikter men vissa är så pass förskjutna åt ena hållet att de betraktas som fullständiga. Tecknet $\ce{<=>}$ används för att visa jämvikt, se följande exempel:
    \begin{equation*}
        \ce{HCl + H2O <=> H3O+ + Cl-}
    \end{equation*}

    \section{Jämviktskonstanten}

Varje kemisk jämvikt har en s.k. jämviktskonstant $K$. Detta beräknas enligt denna formel\footnote{Se s. 42--48 samt uppgift 3:1--3:3} ($n_{prod} = \text{antal produkter och } n_{reakt} = \text{antal reaktanter}$):
\begin{gather*}
    \label{eq:jmvkonstant}
    K = \frac{\prod_{n=1}^{n_{prod}}[\mathrm{produkt}_n]}{\prod_{n=1}^{n_{reakt}}[\mathrm{reaktant}_n]} \\
    \text{alltså\ldots} \\
    K = \frac{[\mathrm{produkt}_1] \cdot [\mathrm{produkt}_2] \dotsm [\mathrm{produkt}_{n_{prod}}]}{[\mathrm{reaktant}_1] \cdot [\mathrm{reaktant}_2] \dotsm [\mathrm{reaktant}_{n_{reakt}}]}
\end{gather*}
$K$ visar alltså förhållandet mellan produkterna av koncentrationerna av produkter och reaktanter. Detta leder även till dessa två till slutsatser:
\begin{align*}
    \text{större } K \Rightarrow \text{mindre reakt. eller mer prod. i jämförelse} \\
    \text{mindre } K \Rightarrow \text{mer reakt. eller mindre prod. i jämförelse} 
\end{align*}
\pagebreak
\begin{exm}
    Vid jämvikt finns det $ \mathrm{0.045\,M} \ \ce{H2O},\ \mathrm{0.005\,M} \ \ce{H2} \text{ och } \mathrm{0.0025\,M} \ \ce{O2}$ i reaktionen
    \begin{equation*}
        \ce{2H2O <=> 2H2 + O2}
    \end{equation*}
    Sätter man in siffrorna får man
    \begin{equation*}
        K = \frac{\mathrm{[H_2O]^2 \cdot [O_2]}}{\mathrm{[H_2O]^2}} \approx 2.78 \cdot 10^{-4} \, \mathrm{M}
    \end{equation*}
    Lägg märke till att vissa koncentrationer är upphöjda till en exponent. Denna exponent är alltid samma som ämnets koefficient i reaktionen. \\ $\ce{2H2O \rightarrow [H2O]^2}$ exempelvis.
\end{exm}

\subsection{Enheten på \textit{K}}

Detta beräknas med en enhetsanalys på koncentrationerna\footnote{Se uppgift 3:4}.

\begin{exm}
    Givet situationen från ovan, sätt in enheter:
    \begin{equation*}
        K \approx 2.78 \cdot 10^{-4} \mathrm{ \left[\frac{M^2 \cdot M}{M^2} = \frac{M^{\cancelto{1}{3}}}{M^{\cancel{2}}} = M\right]}
    \end{equation*}
\end{exm}

\subsection{Räkna på \textit{K}}

Du ska kunna räkna ut $K$ för en viss reaktion utifrån ett fåtal substansmängder eller koncentrationer\footnote{Se s. 48--49 samt uppgift 3:7}.

\begin{exm}
    Titta på exemplet i denna tabell ($C_0$ är koncentration från början och $C_{jmv}$ är koncentration vid jmv.):
    
    \begin{table}[H]
        \centering
        \begin{tabular}{|c|c|>{\centering\arraybackslash}m{31.5pt}|c|} 
        \cline{2-4}
        \multicolumn{1}{l|}{} & \multicolumn{3}{c|}{$\ce{\hspace{7pt} A \hspace{7pt} + \hspace{7pt} B \hspace{2pt} <=> \hspace{2pt} AB} \hspace{15pt}$}  \\ 
        \hline
        $C_0$                 & $x$   & $x$   & $0$                       \\ 
        \hline
        $\Delta C$            & $-y$  & $-y$  & $+y$                      \\ 
        \hline
        $C_{jmv}$             & $x-y$ & $x-y$ & $y$                       \\
        \hline
        \end{tabular}
    \end{table}
    
    \noindent vilket ger att
    
    \begin{equation*}
        K = \frac{[\mathrm{AB}]}{\mathrm{[A] \cdot [B]}} = \frac{y}{(x-y)^2} \, \mathrm{\left[\frac{M}{M^2}=M^{-1}\right]}
    \end{equation*}
    
    Notera att förhållendet mellan $\Delta C$ hos de olika ämnen är densamma som deras koefficient i rekationen så följande gäller i mer komplexa fall:
    
    \begin{table}[H]
        \centering
        \begin{tabular}{|c|c|>{\centering\arraybackslash}m{31.5pt}|c|} 
        \cline{2-4}
        \multicolumn{1}{l|}{} & \multicolumn{3}{c|}{$\ce{\hspace{10pt} 2A \hspace{7pt} + \hspace{9pt} B \hspace{1.5pt} <=> \hspace{2pt} A2B} \hspace{15pt}$}  \\ 
        \hline
        $C_0$                 & $z$   & $x$   & $0$                       \\ 
        \hline
        $\Delta C$            & $-2y$  & $-y$  & $+y$                      \\ 
        \hline
        $C_{jmv}$             & $z-2y$ & $x-y$ & $y$                       \\
        \hline
        \end{tabular}
    \end{table}
    
    \begin{equation*}
        K = \frac{[\mathrm{AB}]}{\mathrm{[A] \cdot [B]}} = \frac{y}{(z-2y) \cdot (x-y)} \, \mathrm{\left[\frac{M}{M^2}=M^{-1}\right]}
    \end{equation*}
\end{exm}
    \setcounter{exm}{0}
    \section{Förskjutning av reaktioner}

I uppgifter behöver man ofta bestämma hur en rekation kommer \emph{förskjutas} eller vilket håll den kommer ''gå mot''. Alla jämvikter vill till slut uppnå det jämviktsförhållande som är givet av deras $K$-värde under givna förhållanden. Om man börjar från ett tillstånd utan jämvikt eller om jämvikten rubbas kommer reaktionen att förskjutas. Detta innebär att antingen mängden reaktanter eller produkter kommer öka eller minska. När antalet produkter ökar jämfört med reaktanterna kallas det att reaktionen förskjuts åt höger och motsatsen kallas förskjutning åt vänster.

\subsection{Reaktionskvoten}

Förhållandet mellan produkter och reaktanter när det inte råder jämvikt beskrivs av \emph{reaktionskvoten} $Q$. Formeln för $Q$ är exakt samma som för $K$. Vid jämvikt är $Q=K$ men övrigt så är den antingen större eller mindre. Det finns två enkla regler angående $Q$-värdet:
\begin{align*}
    &Q > K \Rightarrow \text{fler produkter eller färre reaktanter} \Rightarrow \\ 
    &\Rightarrow \text{förskjutning åt vänster} \\
    &Q < K \Rightarrow \text{färre produkter eller fler reaktanter} \Rightarrow \\ 
    &\Rightarrow \text{förskjutning åt höger}
\end{align*}

\subsection{Tillskott av ämnen}
\subsubsection[Reaktanter]{Tilskott av reaktanter}
Om det finns en reaktion i jämvikt och fler reaktanter läggs till kommer $\prod^{n_{reakt}}_{n=1}[\mathrm{reaktant}_n]$ öka vilket innebär att $Q$ kommer minska (se \cref{eq:jmvkonstant}). Enligt definitionen ovan kommer reaktionen gå åt höger.
\subsubsection[Produkter]{Tillskott av produkter}
Om en reaktion i jämvikt får ett tillskott av produkter kommer $\prod^{n_{prod}}_{n=1}[\mathrm{produkt}_n]$ att öka vilket innebär att $Q$ kommer öka (se \cref{eq:jmvkonstant}). Enligt definitionen ovan kommer reaktionen gå åt vänster.
    \setcounter{exm}{0}
    \input{chapters/reaktionskvot.tex}
\end{document}