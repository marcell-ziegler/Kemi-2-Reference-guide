\section{Jämviktskonstanten}

Varje kemisk jämvikt har en s.k. jämviktskonstant $K$. Detta beräknas enligt denna formel\footnote[1]{Se s. 42--48 samt uppgift 3:1--3:3} ($n_{prod} = \text{antal produkter och } n_{reakt} = \text{antal reaktanter}$):
\begin{gather*}
    K = \frac{\prod_{n=1}^{n_{prod}}[\mathrm{produkt}_n]}{\prod_{n=1}^{n_{reakt}}[\mathrm{reaktant}_n]} \\
    \text{alltså\ldots} \\
    K = \frac{[\mathrm{produkt}_1] \cdot [\mathrm{produkt}_2] \dotsm [\mathrm{produkt}_{n_{prod}}]}{[\mathrm{reaktant}_1] \cdot [\mathrm{reaktant}_2] \dotsm [\mathrm{reaktant}_{n_{reakt}}]}
\end{gather*}
$K$ visar alltså förhållandet mellan produkterna av koncentrationerna av produkter och reaktanter. Detta leder även till dessa två till slutsatser:
\begin{align*}
    \text{större } K \Rightarrow \text{mindre reakt. eller mer prod. i jämförelse} \\
    \text{mindre } K \Rightarrow \text{mer reakt. eller mindre prod. i jämförelse} 
\end{align*}
\pagebreak
\begin{exm}
    Vid jämvikt finns det $ \mathrm{0.045\,M} \ \ce{H2O},\ \mathrm{0.005\,M} \ \ce{H2} \text{ och } \mathrm{0.0025\,M} \ \ce{O2}$ i reaktionen
    \begin{equation*}
        \ce{2H2O <=> 2H2 + O2}
    \end{equation*}
    Sätter man in siffrorna får man
    \begin{equation*}
        K = \frac{\mathrm{[H_2O]^2 \cdot [O_2]}}{\mathrm{[H_2O]^2}} \approx 2.78 \cdot 10^{-4} \, \mathrm{M}
    \end{equation*}
    Lägg märke till att vissa koncentrationer är upphöjda till en exponent. Denna exponent är alltid samma som ämnets koefficient i reaktionen. \\ $\ce{2H2O \rightarrow [H2O]^2}$ exempelvis.
\end{exm}

\subsection{Enheten på konstanten}

Detta beräknas med en enhetsanalys på koncentrationerna.

\begin{exm}
    Givet situationen från ovan, sätt in enheter:
    \begin{equation*}
        K \approx 2.78 \cdot 10^{-4} \mathrm{ \left[\frac{M^2 \cdot M}{M^2} = \frac{M^{\cancelto{1}{3}}}{M^{\cancel{2}}} = M\right]}
    \end{equation*}
\end{exm}