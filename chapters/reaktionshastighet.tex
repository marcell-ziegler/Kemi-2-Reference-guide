\section{Påverkande faktorer}

\subsection{Bindningar}
Fria joner reagerar nästa omedelbart, exempelvis i fällningar. När detta sker är reaktionen \emph{momentan}. Fri joner har inte några bindningar som måste brytas innan reaktionen kan ske vilket gör det snabbare. Molekyler kommer alltid att reagera långsammare då det tar tid och energi att bryta deras bindningar för reaktionen.

\subsection{Kontaktyta}
Jo större ytan som reaktionen sker på är desto snabbare kommmer den att gå. Detta beror på att de reagerande ämnena måste kollidera för att reaktionen ska ske. Ju mer yta att kollidera på desto fler kollisioner alltså desto snabbare reaktion.

\subsection{Aggregationstillstånd}
Aggregationstillståndet, eller i detta fall hur fritt partiklarna rör sig, kommer ha en stor effekt på reaktionshastigheten. En gas kommer ju ha högst rörlighet, sedan vätskor och sist fasta ämnen. Ju mer partiklarna rör sig desto fler chanser kommer de få att kollidera med varandra och desto snabbare kommer reaktionen att gå.

\subsection{Katalysatorer}
\label{sec:katalysator}
En katalysator är ett ämne som gör en reaktion snabbare eller möjliggör en reaktion som annars är omöjlig utan att själv förbrukas. Den gör detta genom att sänka aktiveringsenergin (\textcolor{red}{\underline{\textbf{\textit{missing reference}}}})av reaktionen. Antingen kommer den att hamna under gränsen för tillgänglig energi vilket möjliggör reaktionen eller så kommer den helt enkelt minska energikravet och öka hastigheten.

\subsection{Koncentration}
En högre koncentration gör partiklarna tätare vilket i sin tur leder till snabbare reaktioner. Detta medför också att tryckförändringar påverkar reaktionshastighet i komprimerbar materia (ex. gaser).